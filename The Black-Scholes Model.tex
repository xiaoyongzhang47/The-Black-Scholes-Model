\ifdefined\isSubfile
\else
  \documentclass{book}
  \usepackage{fullpage}
\usepackage{physics}
\usepackage{enumitem}
\usepackage{graphicx}
\usepackage{float}
\usepackage{dsfont}
\usepackage{amsthm}
\usepackage{amssymb}
\usepackage{bm}
\usepackage[unicode]{hyperref}
\usepackage{color}
\usepackage{subcaption}
\usepackage{tikz}
\usepackage{tocloft}
\newcommand*\circled[1]{\tikz[baseline=(char.base)]{
            \node[shape=circle,draw,inner sep=2pt] (char) {#1};}}

\usepackage[backend=bibtex]{biblatex}
\addbibresource{ref.bib}

\newcommand{\listproblemname}{List of Problems}
\newlistof[section]{problem}{prob}{\listproblemname}
\newcommand{\problem}[1]{%
  \refstepcounter{problem}
  \par\noindent\textbf{Problem \theproblem. #1}
  \addcontentsline{prob}{problem}
    {\protect\numberline{\theproblem}\ \ \ \ #1}\par}

\newcommand{\listtheoremmname}{List of Theorems}
\newlistof[section]{theorem}{theo}{\listtheoremmname}
\newcommand{\theorem}[1]{%
  \refstepcounter{theorem}
  \par\noindent\textbf{Theorem \thetheorem. #1}
  \addcontentsline{theo}{theorem}
    {\protect\numberline{\thetheorem}\ \ \ \ #1}\par}

\newcommand{\listconclusionmname}{List of Useful Conclusions}
\newlistof[section]{conclusion}{conc}{\listconclusionmname}
\newcommand{\conclusion}[1]{%
  \refstepcounter{conclusion}
  \par\noindent\textbf{Conclusion \theconclusion. #1}
  \addcontentsline{conc}{conclusion}
    {\protect\numberline{\theconclusion}\ \ \ \ #1}\par}

  \newcommand{\listpropositionname}{List of Propotisitons}
  \newlistof[section]{proposition}{prop}{\listpropositionname}
  \newcommand{\proposition}[1]{%
    \refstepcounter{proposition}
    \par\noindent\textbf{Proposition \theproposition. #1}
    \addcontentsline{prop}{proposition}
      {\protect\numberline{\theproposition}\ \ \ \ #1}\par}

\DeclareMathOperator{\sgn}{sgn}
\DeclareMathOperator{\varr}{\mathbb Var}
\DeclareMathOperator{\cov}{\mathbb Cov}
\DeclareMathOperator{\corr}{corr}
\newcommand{\overbar}[1]{\mkern 1.5mu\overline{\mkern-1.5mu#1\mkern-1.5mu}\mkern 1.5mu}

  \begin{document}
\fi

\chapter{The Black-Scholes Model}
Let's review some common derivative knowledge first. 
\begin{enumerate}
    \item European option: the payoff is path independent of the underlying asset (in this chapter, we will simply call it the stock).
    \item Call options have payoff $(S_T-K)^+ = \max(S_T-K,0)$ at maturity.
    \item Put options have payoff $(K-S_T)^+ = \max(K-S_T,0)$ at maturity.
\end{enumerate}
Since the put-call parity relates the price of a put option and a call option, we would focus on call options in this chapter.

We assume that the stock price $S_t$ follows a Geometric Brownian Motion (GBM) under the real probability measure $\mathbb{P}$, 
\begin{align}
    \dd S_t &= \mu S_t \dd t +\sigma S_t \dd B_t^\mathbb{P},
    \label{eqn:BSM:stock-GBM-P}
\end{align}
where $B_t^\mathbb{P}$ is the standard Brownian motion (Wiener process) under measure $\mathbb{P}$, $\mu$ is the drift rate of $S$ (annualized), and $\sigma$ is the standard deviation of the stock's return. This equation is understood under the discrete limit
\begin{align}
    S_{t+\Delta t} -S_t &= \underbrace{\mu S_t \Delta t}_{\text{drift term}} + \underbrace{\sigma S_t (B_{t+\Delta t}^\mathbb{P} - B_{t}^\mathbb{P})}_{\text{diffusion term, } \sim \mathcal{N}(0,\Delta t)}.
    \label{eqn:BSM:stock-GBM-Q-discrete}
\end{align}
In many cases, it is easier to transform this to the risk neutral measure $\mathbb{Q}$
\begin{align}
    \dd S_t &= r S_t \dd t +\sigma S_t \dd B_t^\mathbb{Q},
    \label{eqn:BSM:stock-GBM-Q}
\end{align}
where $B_t^\mathbb{Q}$ is the standard Brownian motion under measure $\mathbb{Q}$, and $r$ is risk free rate. \textcolor{blue}{write something on the measure change, the Girsanov’s theorem}

The value of a call option $C(t,S)$ at a specific time $t$ is $C(t, S_t)$. Traditionally, there are two approaches to derive the Black-Scholes Model.
\begin{enumerate}
    \item PDE: Construct a PDE for $C(t,S)$, which can be reduce to a heat equation, then solve the heat equation to get the solution.
    \item Probabilistic: the fair price of a call option $C(t, S_t)$ should be the payoff $(S_T-K)^+$ at maturity $T$ discounts to the current time $t$.
    \begin{align}
        C(t, S_t) = e^{-r (T-t)} \mathbb{E}^{\mathbb{Q}}[(S_T-K)^+ | S_t = S].
        \label{eqn:BSM:call-as-expect}
    \end{align}
    Here we note that the conditional expectation is conditioned on $S_t$ is due to the Markovian of $S_t$.
\end{enumerate}

\section{The PDE Approach}

\subsection{The Black-Scholes PDE}

Let's start by applying Itô’s lemma to a call option $C(t, S_t)$ under the risk-neutral measure $\mathbb{Q}$, we find
\begin{align}
    \dd C(t, S_t) &= \pdv{C}{t} \dd t + \pdv{C}{S} \dd S_t + \frac{1}{2}{\pdv[2]{C}{S}}{\dd} [S]_t, \notag\\
    &= \pdv{C}{t} \dd t + {\pdv{C}{S}} (rS_t \dd t+\sigma S_t \dd B_t^\mathbb{Q}) + \frac{1}{2}{\pdv[2]{C}{S}}\sigma^2 S_t^2 \dd t, \notag\\
    &= \left(\pdv{C}{t} + rS_t {\pdv{C}{S}} + \frac{1}{2} \sigma^2 S_t^2 {\pdv[2]{C}{S}} \right) \dd t + \sigma S_t {\pdv{C}{S}} \dd B_t^\mathbb{Q},
\end{align}
where $[S]_t = \int_0^t (\dd S_u)^2$ is the quadratic variation of $S$. 

Consider a self-financing portfolio $\Pi_t = C(t, S_t) - \Delta_t S_t$, which is required to be risk-free, i.e. $\dd \Pi_t = r \Pi_t \dd t$. 
From 
\begin{align}
    \dd \Pi_t &= \dd C(t, S_t) - \Delta_t \dd S_t, \notag\\ 
    \label{eqn:BS:dPI}
    &= \left(\sigma S_t {\pdv{C}{S}} - \Delta_t \sigma S_t \right)\dd B_t^\mathbb{Q} + 
    \left(
    \pdv{C}{t} + rS_t {\pdv{C}{S}} + \frac{1}{2} \sigma^2 S_t^2 {\pdv[2]{C}{S}} - \Delta_t r S_t
    \right)
    \dd t,
\end{align}
eliminating the stochastic term determines $\Delta_t$
\begin{align}
    \sigma S_t {\pdv{C}{S}} - \Delta_t \sigma S_t  = 0,\quad \Rightarrow\quad \Delta_t = {\pdv{C}{S}}.
\end{align}
$\Delta_t$ is known as the hedge ratio or Greek Delta.
Substituting $\Delta_t$ back into the drift term in $\dd \Pi_t$~(\ref{eqn:BS:dPI}) gives
\begin{align*}
    \pdv{C}{t} + \frac{1}{2} \sigma^2 S_t^2 {\pdv[2]{C}{S}} = r \underbrace{\left(C(t, S_t) - \pdv{C}{S} S_t\right)}_{\Pi_t}.
\end{align*}
Rearranging then yields the classic Black-Scholes PDE
\begin{align}
    \label{eqn:BS:Black-Scholes-Equation}
    \pdv{C}{t} + r S \pdv{C}{S} +\frac{1}{2}\sigma^2 S^2 \pdv[2]{C}{S} = rC.
\end{align}
The boundary conditions for the Black-Scholes PDE are
\begin{align}
    C(T,S) = (S-K)^+,\quad C(t,0) = 0.
\end{align}

\subsection{The Black-Scholes PDE and Heat Equations}

It is well known that the Black-Scholes equation can be solved by reducing to a standard heat equation. We will verify it in this section. To solve the Black-Scholes PDE, we introduce the log-price $x$ and time variable $\tau$
\begin{align}
    \label{eqn:BS-heat:var-tf-x-S-and-t-tau}
    x = \log\frac{S}{K},\quad \tau = \frac{\sigma^2}{2}(T - t),
\end{align}
so that
\begin{align}
    \label{eqn:BS-heat:var-tf-S-x-and-t-tau}
    S = K e^x, \quad t = T - \frac{2\tau}{\sigma^2},
\end{align}
and define
\begin{align}
    \label{eqn:BS-heat:var-tf-C-v}
    C(t, S) = Kv(x,\tau).
\end{align}

To express the Black-Scholes PDE in terms of $v, x, \tau$, we first compute the time derivative
\begin{align*}
    \pdv{v}{\tau} = \frac{1}{K}\pdv{C}{\tau} 
    = \frac{1}{K}\pdv{C}{t}\pdv{t}{\tau}
    = \frac{-2}{K\sigma^2}\pdv{C}{t},
\end{align*}
that is 
\begin{align}
    \label{eqn:BS-heat:dC_dt}
    \pdv{C}{t} = -\frac{K\sigma^2}{2} \pdv{v}{\tau}.
\end{align}
Next, compute the first order spatial derivative
\begin{align*}
    \pdv{v}{x} = \frac{1}{K} \pdv{C}{x}
    = \frac{1}{K} \pdv{C}{S}\pdv{S}{x}
    = e^{x} \pdv{C}{S},
\end{align*}
that is
\begin{align}
    \label{eqn:BS-heat:dC_dS}
    \pdv{C}{S} = e^{-x} \pdv{v}{x}.
\end{align}
And compute the second order derivative,
\begin{align*}
    \pdv[2]{v}{x} = \pdv{}{x}\left(e^{x} \pdv{C}{S}\right)
    = e^{x}\pdv{C}{S} + e^{x} \pdv{}{S}\left( \pdv{C}{x} \right)
    = e^{x}\pdv{C}{S} + K e^{2x} \pdv[2]{C}{S},
\end{align*}
that is 
\begin{align}
    \label{eqn:BS-heat:d2C_dS2}
    \pdv[2]{C}{S} = \frac{1}{K}e^{-2x}\left(\pdv[2]{v}{x} - \pdv{v}{x}\right).
\end{align}
Substituting Eqs.~(\ref{eqn:BS-heat:dC_dt}), (\ref{eqn:BS-heat:dC_dS}) and (\ref{eqn:BS-heat:d2C_dS2}) into the Black-Scholes equation (\ref{eqn:BS:Black-Scholes-Equation}), and use the variable transformation Eqs.~(\ref{eqn:BS-heat:var-tf-S-x-and-t-tau}) and (\ref{eqn:BS-heat:var-tf-C-v}) we get
\begin{align}
    -\frac{K\sigma^2}{2} \pdv{v}{\tau} + r K \pdv{v}{x} +\frac{1}{2}\sigma^2 K\left(\pdv[2]{v}{x} - \pdv{v}{x}\right) = rKv &, \notag\\
    \pdv{v}{\tau} - \left(\frac{2r}{\sigma^2} - 1\right)  \pdv{v}{x} - \pdv[2]{v}{x} = - \frac{2r}{\sigma^2}v &, \notag\\
    \label{eqn:BS-heat:PDE-of-v}
    \pdv{v}{\tau} = \pdv[2]{v}{x} + (\lambda - 1) \pdv{v}{x} - \lambda v &, \quad \underline{\lambda = \frac{2r}{\sigma^2},}
\end{align}
with boundary condition
\begin{align}
    v(x,0) = \max(e^{x} - 1, 0).
\end{align}

The equation~(\ref{eqn:BS-heat:PDE-of-v}) still contains a first‐order spatial derivative and a zeroth‐order (“linear”) term. To reduce it to the standard heat equation, set $v(x,\tau) = \exp(Ax + B\tau)\theta(x,\tau)$, applying the method of undetermined coefficients to choose $A$ and $B$ so that both the $\theta$ and the $\partial_x\theta$ vanish. The derivatives are calculated as
\begin{align}
    \label{eqn:BS-heat:dv_dtau}
    \pdv{v}{\tau} &= B \exp(Ax + B\tau)\theta + \exp(Ax + B\tau)\pdv{\theta}{\tau},\\
    \label{eqn:BS-heat:dv_dx}
    \pdv{v}{x} &= A \exp(Ax + B\tau)\theta + \exp(Ax + B\tau)\pdv{\theta}{x},\\
    \label{eqn:BS-heat:d2v_dx2}
    \pdv[2]{v}{x} &= A^2 \exp(Ax + B\tau)\theta + 2 A \exp(Ax + B\tau)\pdv{\theta}{x} + \exp(Ax + B\tau)\pdv[2]{\theta}{x}.
\end{align}
Plugging Eqs.~(\ref{eqn:BS-heat:dv_dtau}), (\ref{eqn:BS-heat:dv_dx}) and (\ref{eqn:BS-heat:d2v_dx2}) to Eq.~(\ref{eqn:BS-heat:PDE-of-v}), we get 
\begin{align}
    B\theta + \pdv{\theta}{\tau}  =  A^2\theta + 2A\pdv{\theta}{x} + \pdv[2]{\theta}{x} + (\lambda - 1) \left(A\theta + \pdv{\theta}{x}\right) - \lambda \theta, \notag\\
    \pdv{\theta}{\tau} = \pdv[2]{\theta}{x} + (2A + \lambda - 1) \pdv{\theta}{x} + \big[A^2 + (\lambda - 1) A - \lambda - B\big] \theta .\notag
\end{align}
To eliminate the $\partial_x\theta$ and $\theta$ terms, $A$ and $B$ must satisfy
\begin{align*}
    \begin{cases}
        2A + \lambda - 1 = 0,\\
        A^2 + (\lambda - 1) A - \lambda - B = 0,
    \end{cases}
\end{align*}
which yields
\begin{align}
    \begin{cases}
        A = -\frac{1}{2} (\lambda - 1),\\
        B = -\frac{1}{4} (\lambda + 1)^2 .
    \end{cases}
\end{align}
Therefore, we have 
\begin{align}
    \label{eqn:BS-heat:var-tf-v-theta}
    v(x, \tau) = \exp[-\frac{1}{2} (\lambda - 1) x -\frac{1}{4} (\lambda + 1)^2 \tau] \theta(x, \tau),
\end{align}
and the Black-Scholes equation reduces to a heat equation
\begin{align}
    \label{eqn:BS-heat:heat-eq-of-theta}
    \pdv{\theta}{\tau} = \pdv[2]{\theta}{x}
\end{align}
with boundary condition
\begin{align}
    \label{eqn:BS-heat:heat-eq-of-theta-bound-cond}
    \theta(x,0) = \exp[\frac{1}{2}(\lambda - 1)x] \max(e^{x} - 1, 0) =  \max\left\{\exp[\frac{1}{2}(\lambda + 1)x] - \exp[\frac{1}{2}(\lambda - 1)x], 0\right\}.
\end{align}

\subsection{The Black-Scholes Formula}

The solution to the heat equation~(\ref{eqn:BS-heat:heat-eq-of-theta}) is given by the integral formula  \textcolor{blue}{write a appendix for heat equation}
\begin{align}
    \label{eqn:BS-form:sol-of-heat-eq}
    \theta(x, \tau) = \frac{1}{2\sqrt{\pi\tau}} \int_{-\infty}^{\infty} \theta(y,0)\exp[-\frac{(x-y)^2}{4\tau}]\dd y,
\end{align}
where $\theta(y,0)$ follows from the boundary condition~(\ref{eqn:BS-heat:heat-eq-of-theta-bound-cond})
\begin{align}
    \label{eqn:BS-form:bond-cond-piece}
    \theta(y,0) &= \max\left\{ \exp(\frac{1}{2}\lambda y) \left[ \exp(\frac{1}{2}y) - \exp(-\frac{1}{2}y)\right], 0\right\}, \notag\\
    &= 
    \left\{
    \begin{aligned}
        &\exp(\frac{1}{2}\lambda y) \left[ \exp(\frac{1}{2}y) - \exp(-\frac{1}{2}y)\right], &&y > 0,\\
        &0, &&y < 0.
    \end{aligned}
    \right.
\end{align}
Substituting the piecewise initial condition from Eq.~(\ref{eqn:BS-form:bond-cond-piece}) into the integral solution~(\ref{eqn:BS-form:sol-of-heat-eq}) yields,
\begin{align}
    \theta(x, \tau) &= \frac{1}{2\sqrt{\pi\tau}} \int_{0}^{\infty}\exp(\frac{1}{2}\lambda y) \left[ \exp(\frac{1}{2}y) - \exp(-\frac{1}{2}y)\right]\exp[-\frac{(x-y)^2}{4\tau}]\dd y, \notag\\
    &= \underbrace{\frac{1}{2\sqrt{\pi\tau}} \int_{0}^{\infty}\exp[\frac{1}{2}(\lambda + 1) y] \exp[-\frac{(x-y)^2}{4\tau}]\dd y}
    _{I_1}
    - 
    \underbrace{\frac{1}{2\sqrt{\pi\tau}} \int_{0}^{\infty}\exp[\frac{1}{2}(\lambda - 1) y] \exp[-\frac{(x-y)^2}{4\tau}]\dd y}
    _{I_2}
    , \notag\\
    &= I_1 - I_2.
    \label{eqn:BS-form:theta-I1-I2}
\end{align}
The solution decomposes into two integrals, $I_1$ and $I_2$, which differ only by the sign in $\lambda \pm 1$.
Since $I_1$ and $I_2$ share the same form, solving one immediately yields the other. To evaluate $I_1$, perform a Gaussian integral via the following substitution
\begin{align}
    z = \frac{y-x}{\sqrt{2\tau}},\quad \dd z = \frac{1}{\sqrt{2\tau}} \dd y.
\end{align}
Applying this substitution to $I_1$ gives
\begin{align}
    I_1 &= \frac{1}{\sqrt{2\pi}}\int_{-\frac{x}{\sqrt{2\tau}}}^\infty 
    \exp[\frac{-z^2}{2}]
    \exp[\frac{1}{2}(\lambda + 1) (\sqrt{2\tau} z + x)]
    \dd z, \notag\\
    &= \frac{1}{\sqrt{2\pi}}\int_{-\frac{x}{\sqrt{2\tau}}}^\infty 
    \exp\left\{\frac{1}{2} 
    \left[-z^2 + (\lambda + 1) \sqrt{2\tau} z - \left((\lambda + 1)\sqrt{\frac{\tau}{2}}\right)^2 
    + (\lambda + 1)^2\frac{\tau}{2} + (\lambda + 1) x\right]
    \right\}\dd z, \notag\\
    &= 
    \exp[\frac{1}{4}(\lambda + 1)^2 \tau + \frac{1}{2} (\lambda + 1) x]
    \frac{1}{\sqrt{2\pi}}
    \int_{-\frac{x}{\sqrt{2\tau}}}^\infty 
    \exp\left\{
     -\frac{1}{2}\left[ z - (\lambda + 1)\sqrt{\frac{\tau}{2}}\right]^2 
    \right\}\dd z, \notag\\
    &= 
    \exp[\frac{1}{2} (\lambda + 1) x + \frac{1}{4}(\lambda + 1)^2 \tau]
    \frac{1}{\sqrt{2\pi}}
    \int_{-\frac{x}{\sqrt{2\tau}} - (\lambda + 1)\sqrt{\frac{\tau}{2}}}^\infty 
    \exp\left\{
     -\frac{1}{2}u ^2 
    \right\}\dd u, \quad \underline{u = z - (\lambda + 1)\sqrt{\frac{\tau}{2}},}\notag\\
    &= 
    \exp[\frac{1}{2} (\lambda + 1) x + \frac{1}{4}(\lambda + 1)^2 \tau]
    N(d_1),
    \quad \underline{d_1 = \frac{x + (\lambda + 1) \tau}{\sqrt{2\tau}}}.
    \label{eqn:BS-form:I1}
\end{align}
Note that for a normal distribution, we use $\mathcal{N}$ for the  probability density function, and $N$ for the cumulative distribution function. Similarly, we obtain
\begin{align}
    I_2 &= \exp[\frac{1}{2} (\lambda - 1) x + \frac{1}{4}(\lambda - 1)^2 \tau]
    N(d_2),
    \quad \underline{d_2 = \frac{x + (\lambda - 1) \tau}{\sqrt{2\tau}}}.
    \label{eqn:BS-form:I2}
\end{align}
Finally, we express the call option $C(t,S)$ in $\theta(x,\tau)$, using Eqs.~(\ref{eqn:BS-heat:var-tf-C-v}) and (\ref{eqn:BS-heat:var-tf-v-theta}). 
Then, substitute $I_1$~(\ref{eqn:BS-form:I1}) and $I_2$~(\ref{eqn:BS-form:I2}) into the integral solution of $\theta(x,\tau)$~(\ref{eqn:BS-form:theta-I1-I2}) yields
\begin{align}
    C(t,S) 
    % &= Kv(x,\tau), \notag\\
    &= K \exp[-\frac{1}{2} (\lambda - 1) x  -\frac{1}{4} (\lambda + 1)^2 \tau] \theta(x, \tau), \notag\\
    &= K \exp[-\frac{1}{2} (\lambda - 1) x -\frac{1}{4} (\lambda + 1)^2 \tau] (I_1 - I_2), \notag\\
    &= K \exp(x) N(d_1) - K \exp(-\lambda \tau) N(d_2). \notag
    % C(t,S) &=  K \exp[\log(S/K)] N(d_1) - K \exp(-\frac{2r}{\sigma^2} \sigma^2(T-t)/2) N(d_2), \notag\\
\end{align}
Replacing $x, \tau, \lambda$ by their definitions in terms of $S, t, r, \sigma$ via Eqs.~(\ref{eqn:BS-heat:var-tf-x-S-and-t-tau}) and (\ref{eqn:BS-heat:PDE-of-v}), we get the Black-Scholes formula for call options,
\begin{align}
    C(t,S) &=  S N(d_1) - K \exp[-r (T-t)] N(d_2).
    \label{eqn:BS-form:call}
\end{align}
with parameters $d_1$ and $d_2$ given by
\begin{align}
    % d_{1,2} &= \frac{\log(S/K) + (\frac{2r}{\sigma^2} \pm 1) \frac{\sigma^2(T-t)}{2}}{\sqrt{2\frac{\sigma^2(T-t)}{2}}}, \notag\\
    d_{1,2} &= \frac{\log(\frac{S}{K}) + \left[r \pm \frac{\sigma^2}{2}\right] (T-t) }{\sigma\sqrt{T-t}}.
    \label{eqn:BS-form:d1-and-d2}
\end{align}

\section{Probabilistic Approach}

Under the risk neutral measure $\mathbb{Q}$, the stock price follows a Geometric Brownian Motion (previously shown in Eq.~(\ref{eqn:BSM:stock-GBM-Q}))
\begin{align*}
    \dd S_t &= r S_t \dd t +\sigma S_t \dd B_t^\mathbb{Q},
\end{align*}
where $r$ is risk free rate. The first step is to solve for $S_t$. Let $A_t = \log S_t$. Applying Itô’s lemma yields
\begin{align}
    \dd A_t &= \frac{1}{S_t} \dd S_t - \frac{1}{2} \frac{1}{S_t^2} {\dd} {[S]_t}, \notag\\
    &= r \dd t +\sigma \dd B_t^\mathbb{Q} - \frac{1}{2} \sigma^2 \dd t, \notag\\
    &= \left(r - \frac{1}{2} \sigma^2\right) \dd t + \sigma  \dd B_t^\mathbb{Q}.
\end{align}
Since the right-hand side is entirely known, we integrate and obtain
\begin{align}
    A_t &= A_0 + \int_0^t \left(r - \frac{1}{2} \sigma^2\right) \dd s + \sigma \int_0^t \dd B_s^\mathbb{Q}, \notag\\
    &=  A_0 + \left(r - \frac{1}{2} \sigma^2\right)t+ \sigma  B_t^\mathbb{Q}, \notag\\
    S_t &= {S_0} \exp \left[ \left(r - \frac{1}{2} \sigma^2\right)t + \sigma  B_t^\mathbb{Q} \right].
\end{align}
Thus, looking from the current time $t$, the solution at the expiration time $T$ is 
\begin{align}
    S_T &= {S_t} \exp \left[ \left(r - \frac{1}{2} \sigma^2\right)(T - t) + \sigma  \left(B_T^\mathbb{Q} - B_t^\mathbb{Q}\right) \right].
\end{align}
To better understand the distribution of $S_T$, we take the logarithm
\begin{align}
    \log S_T &= \underbrace{\log {S_t} + \left(r - \frac{1}{2} \sigma^2\right)(T - t)}_{\text{deterministic value}} + \sigma \underbrace{\left(B_T^\mathbb{Q} - B_t^\mathbb{Q}\right)}_{\text{r.v.} \sim \mathcal{N}(0,  T - t)},
\end{align}
that is, $\log S_T$ is a normal distribution, hence $S_t$ is a log-normal distribution. Define $\tau = T - t$, the time to maturity, and let $Y\sim \mathcal{N}(0,1)$ be the standard normal distribution. We have a cleaner form for $S_T$
\begin{align}
    S_T &= {S_t} \exp \left[ \left(r - \frac{1}{2} \sigma^2\right)\tau + \sigma  \sqrt{\tau} Y \right].
\end{align}
Recall that the price for call options is the conditional expectation of $(S - K)^+$ at maturity discounted to the current time $t$, i.e., Eq.~(\ref{eqn:BSM:call-as-expect})
\begin{align*}
    C(t, S_t) = e^{-r \tau} \mathbb{E}^{\mathbb{Q}}[(S_T-K)^+ | S_t = S].
\end{align*}
We perform the following calculation
\begin{align*}
    C(t,S_t) &= \exp(-r\tau) \mathbb{E}^\mathbb{Q}\left[ \left({S_t} \exp \left\{ \left(r - \frac{1}{2} \sigma^2\right)\tau + \sigma  \sqrt{\tau} Y \right\} - K \right)^+ \Bigg| S_t = S\right], \notag\\
    &=  \exp(-r\tau) \mathbb{E}^\mathbb{Q}\left[ \left(S \exp \left\{ \left(r - \frac{1}{2} \sigma^2\right)\tau + \sigma  \sqrt{\tau} Y \right\} - K \right)^+ \right], \notag\\
    &= \exp(-r\tau) \int_{-\infty}^{+\infty} \left(S \exp \left\{ \left(r - \frac{1}{2} \sigma^2\right)\tau + \sigma  \sqrt{\tau} y \right\} - K \right)^+  \frac{1}{\sqrt{2\pi}}\exp({-\frac{y^2}{2}}) \dd y. \notag
\end{align*}
To get rid of the positive-part operator $()^+$, we calculate the critical point of $y$
\begin{align}
    S \exp \left\{ \left(r - \frac{1}{2} \sigma^2\right)\tau + \sigma  \sqrt{\tau} y \right\} - K &=0, \notag\\
    \left(r - \frac{1}{2} \sigma^2\right)\tau + \sigma  \sqrt{\tau} y &= \log \frac{K}{S}, \notag\\
    y &= \underbrace{\frac{\log \frac{K}{S} - \left(r - \frac{1}{2} \sigma^2\right)\tau} {\sigma \sqrt{\tau}}}_{ =\, l}.
\end{align}
With this information, we proceed with our calculation
\begin{align}
    C(t,S_t) &= \exp(-r\tau) \int_{l}^{+\infty} \left(S \exp \left\{ \left(r - \frac{1}{2} \sigma^2\right)\tau + \sigma  \sqrt{\tau} y \right\} - K \right) \frac{1}{\sqrt{2\pi}}\exp({-\frac{y^2}{2}}) \dd y, \notag\\
    &= S \int_{l}^{+\infty}  \exp (- \frac{1}{2} \sigma^2 \tau + \sigma \sqrt{\tau} y) \frac{1}{\sqrt{2\pi}}\exp({-\frac{y^2}{2}}) \dd y - \exp(-r\tau) KN(-l), \notag\\
    &= S \int_{l}^{+\infty} \frac{1}{\sqrt{2\pi}}\exp[{-\frac{1}{2}} \left(y^2 - 2\sigma  \sqrt{\tau} y + \big(\sigma\sqrt{\tau}\big)^2\right)] \dd y - K \exp(-r\tau) N(-l), \notag \\
    &= S \int_{l - \sigma\sqrt{\tau}}^{+\infty} \frac{1}{\sqrt{2\pi}}\exp(-\frac{1}{2} u^2) \dd u - K \exp(-r\tau) N(-l), \quad  \underline{u = y - \sigma\sqrt{\tau},} \notag\\
    &= S N (\sigma\sqrt{\tau} - l) - K \exp(-r\tau) N(-l), \notag \\
    &= S N (d_1) - K \exp(-r\tau) N(d_2), 
    \label{eqn:BS-Exp:call}
\end{align}
where in the last step, we simply verify that
\begin{align*}
    \sigma\sqrt{\tau} - l &= \frac{\sigma^2 \tau + \log \frac{S}{K} + \left(r - \frac{1}{2} \sigma^2\right)\tau} {\sigma \sqrt{\tau}} = \frac{\log \frac{S}{K} + \left(r + \frac{1}{2} \sigma^2\right)\tau} {\sigma \sqrt{\tau}} = d_1, \\
    -l &= \frac{\log \frac{S}{K} + \left(r - \frac{1}{2} \sigma^2\right)\tau} {\sigma \sqrt{\tau}} = d_2.
\end{align*}
This result Eq.~(\ref{eqn:BS-Exp:call}) calculated from expectation matches the result Eq.~(\ref{eqn:BS-form:call}) from solving the Black-Scholes PDE.

\ifdefined\isSubfile
\else
  % \bibliographystyle{alpha}
  % \bibliography{ref}
  \end{document}
\fi
